% !TEX options=--shell-escape

\documentclass{article}
\usepackage[T1]{fontenc}

\usepackage{titlesec}
\usepackage{hyperref}

\usepackage[a4paper, margin=2.5cm, headsep=0pt]{geometry}

\usepackage{tgadventor}
\renewcommand{\familydefault}{\sfdefault}

\usepackage{graphicx}
\usepackage{titlepic}
\usepackage[skip=5pt]{caption}

\usepackage[ddmmyyyy]{datetime}
\usepackage[section]{placeins}
\usepackage{enumitem}

\titleformat{\chapter}{\normalfont\huge}{\bf\thechapter.}{20pt}{\huge\bf}
\titlespacing{\chapter}{0pt}{12pt plus 4pt minus 2pt}{8pt plus 2pt minus 2pt}

\hypersetup{
    colorlinks,
    citecolor=black,
    filecolor=black,
    linkcolor=black,
    urlcolor=blue
}

\newdate{creation}{04}{02}{2022}

\begin{document}
\begin{titlepage}
    \centering
    \vfill
    \bfseries\Huge{Pacenstein\\\large{- Retrospectieven -}}\\
    \normalfont\normalsize\displaydate{creation}
    \vfill

    \includegraphics[width=\textwidth]{../res/pacenstein.png}
    \vfill
    \large{
        Lennard Duinkerken\\
        Emma Raijmakers\\
        Daan Roth\\
        Jarno Bröcker
    }
    \vfill
\end{titlepage}

\newpage
\tableofcontents
\newpage

\section{Week 1} % (fold)
\label{sec:week_1}

\subsection{Daan} % (fold)
\label{sub:daan}
- Zelfde indentatie gaan gebruiken, allemaal spaces in plaats van tabs.\\
+ Iedereen neemt goed initiatief door zelf punten uit de backlog te kiezen.\\
+ De makefile is flexibel genoeg dat de verschillende werkwijzen en besturing systemen van het team correct functioneren.\\
+ Github omgeving is goed ingericht, makkelijk issues aanmaken en sluiten gebaseerd op on kanban board.
% subsection daan (end)

\subsection{Emma} % (fold)
\label{sub:emma}
+ Ik vond dat er in deze sprint een fijne samenwerking was tussen alle teamgenoten.\\
+ Ik had het gevoel dat ik voor al mijn vragen bij mijn teamgenoten terecht kom.\\
+ Er was een duidelijke planning en overzicht van alle taken.\\
- Door corona hebben we volledig online samengewerkt. Als het weer kan, zou ik graag ook fysiek aan de game willen werken.
% subsection emma (end)

\subsection{Lennard} % (fold)
\label{sub:lennard}
+ Iedereen heeft goed in de vingers wat ze moeten doen.\\
+ Vragen werden snel beantwoord.\\
+ Eerlijke verdeling van taken.\\
+ Iedereen doorgaans op tijd aanwezig.\\
- Merge conflicten galore, maar werden goed opgelost en heeft niet echt problemen opgeleverd.
% subsection lennard (end)

\subsection{Jarno} % (fold)
\label{sub:jarno}
+ Goed overzicht gehouden met behulp van github (project planner).\\
+ Er werd samen ervoor gezorgd dat iedereen wat te doen had. Iedereen heeft daardoor ook ongeveer evenveel gedaan.\\
+ Er werd om hulp gevraagd waar nodig en die hulp werd ook gegeven.\\
+ Als iemand op een bepaald moment niet kon, werd dat van te voren aangegeven in een ‘beschikbaarheids bestand’.\\
- Door Corona besmettingen alleen online kunnen werken.
% subsection jarno (end)
% section week_1 (end)

\section{Week 2} % (fold)
\label{sec:week_2}

\subsection{Daan} % (fold)
\label{sub:daan}
+ Het samenwerken met git ging deze week al een stuk beter. Minder merge conflicten.\\
+ De taakverdeling was nog steeds goed.\\
+ Iedereen gebruikte dezelfde code standaard.\\
- Mogelijk moeten we wat features inkorten omdat we dit te veel hadden onderschat.
% subsection daan (end)

\subsection{Emma} % (fold)
\label{sub:emma}
+ Tijdens deze sprint is er initiatief getoont om de minpunten van de vorige sprint te verbeteren. We hebben deze week een keer af kunnen spreken op de HU om aan de game te werken.\\
+ De samenwerking was, net als bij de vorige sprint, goed.\\
- We zijn iets te enthousiast geweest met de functionaliteiten die we toe willen voegen aan onze game. Er staan nog best veel opdrachten op de planning en ik verwacht dat we ze niet allemaal af zullen krijgen.
% subsection emma (end)

\subsection{Lennard} % (fold)
\label{sub:lennard}
+ Nog steeds goede taakverdeling en samenwerking.\\
+ Minder merge conflicten vanwege betere communicatie.
% subsection lennard (end)

\subsection{Jarno} % (fold)
\label{sub:jarno}
+ Samenwerking bleef erg goed gaan.\\
+ Code standaard was bij iedereen hetzelfde.\\
- Doordat we lazertag ook nog in moesten halen hebben we het erg druk.\\
- Weinig op school.
% subsection jarno (end)
% section week_2 (end)

\section{Week 3} % (fold)
\label{sec:week_3}

\subsection{Daan} % (fold)
\label{sub:daan}
+ Op een kleine miscommunicatie over thuis of op school werken na ging de communicatie erg goed.\\
+ De laatste afrondingen gingen goed. We hebben een complete speelbare game.\\
- We hebben de laatste avond nog wel aan het project moeten werken, wat misschien een tikje laat is.\\
Over het algemeen is dit project best goed gegaan. Er waren punten die erg lastig waren, en nog niet perfect functioneren zoals de raycaster. Maar we hebben een volledige game die leuk is om te spelen.
% subsection daan (end)

\subsection{Emma} % (fold)
\label{sub:emma}
+ Tijdens deze sprint hebben we prioriteiten gesteld over de opdrachten die we nog moesten maken voor de game.\\
+ We hebben een game kunnen neerzetten waarvan de meeste functionaliteiten van Pacman zijn toegevoegd en de gameplay werkt. Maar nog belangrijker is dat het eindresultaat van de game iets is waar we trots op kunnen zijn.\\
% subsection emma (end)

\subsection{Lennard} % (fold)
\label{sub:lennard}
- Druk met lasertag, dus ik heb weinig tijd besteed aan de game.\\
+ De overige teamleden hebben uitstekend werk geleverd.\\
+ Elke dag op school samenwerken is beter dan via Discord.\\
Al met al is dit project goed en relatief soepel verlopen. Het is de afgelopen 3 weken erg druk geweest vanwege de twee projecten die liepen, in plaats van een project tegelijk.
% subsection lennard (end)

\subsection{Jarno} % (fold)
\label{sub:jarno}
+ De hele week op school gewerkt! (was er fijn!).\\
+ We zijn erg ver gekomen met de game! Echt een goede eindsprint gemaakt.\\
- Door de lasergame hadden we het wat drukker… (toch een mooie game gemaakt).\\
Ik ben erg tevreden met dit project. We hebben een mooie game gemaakt, die ook nog eens erg unique was. Het was een goede samenwerking en ik heb er vertrouwen in dat dit een mooi cijfer wordt!
% subsection jarno (end)
% section week_3 (end)

\end{document}
